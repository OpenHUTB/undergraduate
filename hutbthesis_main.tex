%!TEX program = xelatex 
% !BIB program = biber
%%%%%%%%%%%%%%%%%%%%%%%%%%%%%%%%%%%%%%%%%%%%%%%%%%
% 载入模版
%
% 载入 hutbthesis.cls文件定义的模板
%%%%%%%%%%%%%%%%%%%%%%%%%%%%%%%%%%%%%%%%%%%%%%%%%%

\documentclass[AutoFakeBold]{hutbthesis}

\addbibresource{content/reference.bib}

%%%%%%%%%%%%%%%%%%%%%%%%%%%%%%%%%%%%%%%%%%%%%%%%%%
% 基本信息
%
% 用户自行输入标题、作者等基本信息
% 都存储在\content\info.tex文件中
%%%%%%%%%%%%%%%%%%%%%%%%%%%%%%%%%%%%%%%%%%%%%%%%%%
%!TEX root = ../hutbthesis_main.tex
% 文章信息(同时也是页眉)
\titlecn{湖南工商大学毕业论文标题}
%\header_title{湖南工商大学毕业论文}
\titleen{Hunan University of Technology and Business Thesis \LaTeX{} Template v2.0}


%\minormajor{通信工程}
%\interestmajor{通信工程}
\author{张三}
\subsupervisor{}
\studentid{2212223334}
\priormajor{人工智能}
\myclass{人工智能2201班}
\supervisor{王五}
\title{讲师}
\department{人工智能与先进计算学院}
\thesisdate{year=2025, month=5}

%以下的对本科生没有用
\clcnumber{TP391} 				% 中图分类号 Chinese Library Classification
\schoolcode{10533}			% 学校代码
\udc{004.9}						% UDC
\academiccategory{学术学位}	% 学术类别



\begin{document}
%%%%%%%%%%%%%%%%%%%%%%%%%%%%%%%%%%%%%%%%%%%%%%%%%%
% 封面绘制
%
% 1.5版本重新编写了封面绘制宏,并用latex使用者更习惯的
% \maketitle代替之前的\makecoverpage
%%%%%%%%%%%%%%%%%%%%%%%%%%%%%%%%%%%%%%%%%%%%%%%%%%
\maketitle

%\declarationzh

% 启用大罗马字母进行编号
\frontmatter
% 设置页眉和页脚

%\include{content/info}

%!TEX root = ../hutbthesis_main.tex

\begin{declarationzh}
	
本人郑重声明:所呈交的本科毕业论文是本人在指导老师的指导下,独立进行研究工作所取得的成果,成果不存在知识产权争议,除文中已经注明引用的内容外,本论文不含任何其他个人或集体已经发表或撰写过的作品成果。
对本文的研究做出重要贡献的个人和集体均已在文中以明确方式标明。
本人完全意识到本声明的法律结果由本人承担。
\\
\\
\\
\\
\\
\\
\\
\\
	
	\vspace{30pt}
	\begin{tabular}{ll}
		%\renewcommand{\arraystretch}{2}
		\makebox[4em][s]{作者签名:} & \makebox[100pt][c]{  } \\
		\\
		\makebox[2em][s]{日期:}	 &
		\makebox[100pt][c]{\qquad 年\quad 月\quad   日 } \\
	\end{tabular}

	
	
\end{declarationzh}

% 授权书
%!TEX root = ../hutbthesis_main.tex

\keywordsen{HUTB\ \ LaTeX\ \ Template}
\begin{authorizationzh}
	
本毕业设计《\qquad \qquad \qquad \qquad \qquad \qquad \qquad \qquad \qquad \qquad \qquad》是本人在校期间所完成学业的组成部分,是在学校教师的指导下完成的。
因此,本人特授权学校可将本毕业设计的全部或部分内容编入有关书籍、数据库保存,可采用复制、印刷、网页制作等方式将设计文本和经过编辑、批注等处理的设计文本提供给读者查阅、参考,可向有关学术部门和国家有关教育主管部门呈送复印件和电子文档。
本毕业设计无论做何种处理,必须尊重本人的著作权,署明本人姓名。

	
	\vspace{30pt}
	\begin{tabular}{ll}
		%\renewcommand{\arraystretch}{2}
		\hspace{240pt} \makebox[4em][s]{设计作者(签字):} & \underline{\makebox[100pt][c]{  }} \\
		\hspace{240pt} \makebox[4em][s]{日\qquad 期:}	 &
		\underline{\makebox[100pt][c]{\qquad 年\quad 月\quad   日 }} \\
	\end{tabular}

	
	
\end{authorizationzh}

%%%%%%%%%%%%%%%%%%%%%%%%%%%%%%%%%%%%%%%%%%%%%%%%%%
% 中文摘要
%
% 存储在\content\abstractzh.tex文件中
%%%%%%%%%%%%%%%%%%%%%%%%%%%%%%%%%%%%%%%%%%%%%%%%%%
\include{content/abstractzh}

%%%%%%%%%%%%%%%%%%%%%%%%%%%%%%%%%%%%%%%%%%%%%%%%%%
% 英文摘要
%
% 存储在\content\abstracten.tex文件中
%%%%%%%%%%%%%%%%%%%%%%%%%%%%%%%%%%%%%%%%%%%%%%%%%%
\include{content/abstracten}


%%%%%%%%%%%%%%%%%%%%%%%%%%%%%%%%%%%%%%%%%%%%%%%%%%
% 目录
%
% 使用重定义的tableofcontents宏绘制目录
% 满足学校的样式要求
%%%%%%%%%%%%%%%%%%%%%%%%%%%%%%%%%%%%%%%%%%%%%%%%%%
\tableofcontents


% 启用数字编号,改为第 x 页  共 x 页格式
\mainmatter

%%%%%%%%%%%%%%%%%%%%%%%%%%%%%%%%%%%%%%%%%%%%%%%%%%
% 正文
%
% 存储在\content\content.tex文件中
%%%%%%%%%%%%%%%%%%%%%%%%%%%%%%%%%%%%%%%%%%%%%%%%%%
% 正文
%\include{content/content}


%!TEX root = ../../csuthesis_main.tex
\chapter{绪论}

这是湖南工商大学学位论文\LaTeX{}模板,下面的文字主要作用为对重构后的模板样式设置进行测试。
测试样例基本覆盖模板设定,包括多级标题的基本样式,段落与缩进距离。

\section{研究背景与意义}

\subsection{研究背景}

\subsubsection{研究意义}

一级标题根据学校提供的Word模板要求,三号黑体居中,上下各空一行,章节号空一个汉字,
并且每一章节单独起一页,章节号格式应使用阿拉伯数字而非中文汉字。

二级标题为小四号黑体,缩进两个汉字。章节号后空一个汉字。

三级标题小四号楷体GB2312,字体包含在项目中,同样缩进两个汉字,章节号后空一个汉字。

四级标题参照本科学术论文设计样式,分项采取(1)、(2)、(3)的序号。

所有标题样式由\cls{undergraduate.cls}模板文件 \cs{ctexset} 进行设置。

\section{国内外研究现状}

\subsection{国内研究现状}

\subsection{国外研究现状}

\subsection{研究述评}


\section{研究内容与创新点}


\section{论文组织架构}

正文字体默认使用小四号宋体,英文为小四号 Times New Romen,各段行首缩进两个汉字

承千年文脉,扬湖湘精神。湖南工商大学坐落在历史文化名城长沙,创建于1949年,享有“经济湘军基地,企业名家摇篮”的盛誉。她是一所院士领衔的涵盖管理学、经济学、工学、理学、法学、文学、艺术学、交叉学科等多学科相互支撑、协调发展、特色鲜明的财经类大学,是湖南省本科一批招生高校、教育部本科教学工作水平评估优秀高校、博士学位授予立项建设单位、“十三五”国家产教融合发展工程应用型本科高校、全国首批百强“深化创新创业教育改革示范高校”、全国高校实践育人创新创业基地、教育部人文社会科学优秀成果奖大满贯高校。

学校拥有一批以中国工程院院士陈晓红为代表,包括国务院学位委员会管理科学与工程学科评议组召集人、国家自然科学基金委员会委员、教育部管理科学与工程类专业教学指导委员会副主任委员、教育部科技委管理学部副主任、国家基础科学中心主任、国家一级重点学科“管理科学与工程”和国家自然科学基金委创新研究群体负责人、教育部“长江学者创新团队”首席教授、国家“万人计划”领军人才、全国文化名家暨“四个一批”人才、国家首批“百千万人才工程”第一层次人选等在内的国家级高层次人才;拥有高级职称教师近500人,具有博士学位教师近800人;引智院士9名、“杰青”“长江”等专家学者和优秀企业家70人;院士团队入选“全国高校黄大年式教师团队”。

英文字体展示如下:

TeX (/tɛx, tɛk/, see below), stylized within the system as TEX, is a typesetting system (or a "formatting system") which was designed and mostly written by Donald Knuth\cite{knuth1984texbook} and released in 1978. TeX is a popular means of typesetting complex mathematical formulae; it has been noted as one of the most sophisticated digital typographical systems.


\subsection{调节字号}

可以使用 \cs{zihao}命令来调节字号。

\begin{tabular}{ll}
  \verb|\zihao{3} | & \zihao{3}  三号字 English \\
  \verb|\zihao{-3}| & \zihao{-3} 小三号 English \\
  \verb|\zihao{4} | & \zihao{4}  四号字 English \\
  \verb|\zihao{-4}| & \zihao{-4} 小四号 English \\
  \verb|\zihao{5} | & \zihao{5}  五号字 English \\
  \verb|\zihao{-5}| & \zihao{-5} 小五号 English \\
\end{tabular}

\subsection{调节字体}

需要说明的是由于学校写作指导要求的字体部分不可在Linux上使用,即便你的写作过程是在Linux或者macOS上完成的,
我们仍\textbf{强烈建议}您在Windows操作系统上编译最终版论文。

中文字体可以使用如下命令来调节。

\begin{tabular}{l l}
  \verb|\songti| & {\songti 宋体} \\
  \verb|\heiti| & {\heiti 黑体} \\
%   \verb|\kaiti| & {\kaiti 楷体}
\end{tabular}


\section{模板主要结构}

本项目模板的主要结构, 如下表所示:
% TODO 进一步完善

\begin{table}[ht]
  \centering
  \begin{tabular}{r|l|l}
    \hline\hline
    \multicolumn{2}{l|}{csuthesis\_main.tex } & 主文档,可以理解为文章入口。                                      \\ \hline
                                                & info.tex   & 作者、文章基本信息 \\ \cline{2-3}
                                                & abstactzh/en.tex    & 中/英文摘要内容 \\ \cline{2-3}
    \raisebox{1em}{content 目录 }          &  subchapters 目录   & 章节内容           \\ \hline
    \multicolumn{2}{l|}{images 目录}         & 用于存放图片文件                                                \\ \hline
    \multicolumn{2}{l|}{hutbthesis.cls }       & 模板入口                         \\ \hline\hline
  \end{tabular}
\end{table}

我们不建议模板使用者更改原有模板的结构,
但如果您确实需要,请务必先充分阅读本模板的使用说明并了解相应的\LaTeX{}模板设计知识。

%!TEX root = ../../csuthesis_main.tex
\chapter{相关理论基础(图表示例)}

\section{图片与布局}

\subsection{插图}

图片可以通过\cs{includegraphics}指令插入,我们建议模板使用者将文章所需插入的图片源问卷放置在 images 目录中,
另外,矢量图片应使用PDF格式,位图照片则应使用JPG格式(LaTeX不支持TIFF格式)。具有透明背景的栅格图可以使用PNG格式。

下面是一个简单的插图示例。

\begin{figure}[hbt]
    \centering
    \includegraphics[width=0.3\linewidth]{hutb_building.png}
    \caption{插图示例}
    \label{f.example}
\end{figure}


如果一个图由多个分图(子图)组成,应通过(a),(b),(c)进行标识并附注在分图(子图下方)。
目前子图标识不居中问题没有解决,预计下个版本修复。

\subsection{横向布局}

模板提供常见的图片布局,比如单图布局\ref{f.example},另外还有横排布局如下:

\begin{figure}[!htb]
    \centering
    \begin{subfigure}[t]{0.24\linewidth}
        \begin{minipage}[b]{1\linewidth}
        \includegraphics[width=1\linewidth]{hutb_building.png}
        \caption{test}
        \end{minipage}
    \end{subfigure}
    \begin{subfigure}[t]{0.24\linewidth}
        \begin{minipage}[b]{1\linewidth}
        \includegraphics[width=1\linewidth]{hutb_eim.png}
        \caption{test}
        \end{minipage}
    \end{subfigure}
    \begin{subfigure}[t]{0.24\linewidth}
        \begin{minipage}[b]{1\linewidth}
        \includegraphics[width=1\linewidth]{hutb_building.png}
        \caption{test}
        \end{minipage}
    \end{subfigure}
    \begin{subfigure}[t]{0.24\linewidth}
        \begin{minipage}[b]{1\linewidth}
        \includegraphics[width=1\linewidth]{hutb_eim.png}
        \caption{test}
        \end{minipage}
    \end{subfigure}
    \caption{图片横排布局示例}
    \label{f.row}
\end{figure}

\section{纵向布局}

纵向布局如图\ref{f.col}

\begin{figure}[!htb]
    \centering
    \begin{subfigure}[t]{0.15\linewidth}
        \captionsetup{justification=centering} %ugly hacks
        \begin{minipage}[b]{1\linewidth}
        \includegraphics[width=1\linewidth]{hutb_building.png}
        \caption{test}
        \end{minipage}
    \end{subfigure}\\
    \begin{subfigure}[t]{0.15\linewidth}
        \captionsetup{justification=centering} %ugly hacks
        \begin{minipage}[b]{1\linewidth}
        \includegraphics[width=1\linewidth]{hutb_eim.png}
        \caption{test}
        \end{minipage}
    \end{subfigure}
    \caption{图片纵向布局示例}
    \label{f.col}
\end{figure}

\section{竖排多图横排布局}

\begin{figure}[!htb]
    \centering
    \begin{subfigure}[t]{0.13\linewidth}
        \captionsetup{justification=centering} 
        \begin{minipage}[b]{1\linewidth}
        \includegraphics[width=1\linewidth]{hutb_building.png} 
        \vspace{-1ex} \vfill
        \includegraphics[width=1\linewidth]{hutb_eim.png}
        \caption{aaa}
        \end{minipage}
    \end{subfigure}
    \begin{subfigure}[t]{0.13\linewidth}
        \captionsetup{justification=centering} 
        \begin{minipage}[b]{1\linewidth}
        \includegraphics[width=1\linewidth]{hutb_eim.png} 
        \vspace{-1ex} \vfill
        \includegraphics[width=1\linewidth]{hutb_building.png}
        \caption{bbb}
        \end{minipage}
    \end{subfigure}
    \caption{图片竖排多图横排布局}
    \label{f.csu_col_row}
\end{figure}

竖排多图横排布局如图\ref{f.csu_col_row}所示。注意看(a)、(b)编号与图关系


\section{横排多图竖排布局}

潮涌湘江阔,鹏翔天地宽。湖南工商大学正以习近平新时代中国特色社会主义思想为指引,秉持“新工科+新商科+新文科”与理科融合发展的思路,努力形成一流的理念、一流的目标、一流的标准、一流的质量、一流的机制,打造创新工商、人文工商、艺术工商、体育工商、数智工商、绿色工商、幸福工商,建设读书求知的好园地,乘高等教育改革奋进的东风,朝着创新型一流工商大学的愿景扬帆远航。

\begin{figure}[!htb]
    \centering
    \begin{subfigure}[t]{0.3\linewidth}
        \captionsetup{justification=centering} 
        \begin{minipage}[b]{1\linewidth}
        \includegraphics[width=0.45\linewidth]{hutb_building.png}
        \includegraphics[width=0.45\linewidth]{hutb_eim.png}
        \caption{}
        \end{minipage}
    \end{subfigure}\\
    \begin{subfigure}[t]{0.3\linewidth}
        \captionsetup{justification=centering} 
        \begin{minipage}[b]{1\linewidth}
        \includegraphics[width=0.45\linewidth]{hutb_eim.png}
        \includegraphics[width=0.45\linewidth]{hutb_building.png}
        \caption{}
        \end{minipage}
    \end{subfigure}
    \caption{图片横排多图竖排布局}
    \label{f.csu_row_col}
\end{figure}

横排多图竖排布局如图\ref{f.csu_row_col}所示。注意看(a)、(b)编号与图关系。

\newpage
%!TEX root = ../../csuthesis_main.tex
\chapter{基于××算法××××(体现人工智能专业的算法)(表格插入示例)}

\begin{table}[htb]
  \centering
  \caption{学校文件里对表格的要求不是很高,不过按照学术论文的一般规范,表格为三线表。}
  \label{T.example}
  \begin{tabular}{llllll}
  \hline
   & A  & B  & C  & D  & E \\
  \hline
1 	& 212 & 414 & 4 		& 23 & fgw	\\
2 	& 212 & 414 & v 		& 23 & fgw	\\
3 	& 212 & 414 & vfwe		& 23 & 嗯	\\
4 	& 212 & 414 & 4fwe		& 23 & 嗯	\\
5 	& af2 & 4vx & 4 		& 23 & fgw	\\
6 	& af2 & 4vx & 4 		& 23 & fgw	\\
7 	& 212 & 414 & 4 		& 23 & fgw	\\

\hline{}
\end{tabular}
\end{table}

\textbf{表格如表\ref{T.example}所示,latex表格技巧很多,这里不再详细介绍。}

潮涌湘江阔,鹏翔天地宽。湖南工商大学正以习近平新时代中国特色社会主义思想为指引,秉持“新工科+新商科+新文科”与理科融合发展的思路,努力形成一流的理念、一流的目标、一流的标准、一流的质量、一流的机制,打造创新工商、人文工商、艺术工商、体育工商、数智工商、绿色工商、幸福工商,建设读书求知的好园地,乘高等教育改革奋进的东风,朝着创新型一流工商大学的愿景扬帆远航。


\newpage




\newpage
%!TEX root = ../../csuthesis_main.tex

\chapter{×××系统设计与实现(公式插入示例)}

\section{X系统测试}

\section{本章小结}

潮涌湘江阔,鹏翔天地宽。湖南工商大学正以习近平新时代中国特色社会主义思想为指引,秉持“新工科+新商科+新文科”与理科融合发展的思路,努力形成一流的理念、一流的目标、一流的标准、一流的质量、一流的机制,打造创新工商、人文工商、艺术工商、体育工商、数智工商、绿色工商、幸福工商,建设读书求知的好园地,乘高等教育改革奋进的东风,朝着创新型一流工商大学的愿景扬帆远航。


\textbf{公式插入示例如公式(\ref{E.example})所示。}

\begin{equation}
	\gamma_{x}=
	\left\{
	\begin{array}{lr}
		0, & {\rm if}~~\;|x| \leq \delta \\
		x, & {\rm otherwise}
	\end{array}
	\right.
	\label{E.example}
\end{equation}


\newpage




%!TEX root = ../../csuthesis_main.tex
\chapter{总结与展望 (引用文献标注)}

文献标注和索引的处理一直是学术写作的一个麻烦事,特别是在word环境下。latex中我们只需要
编辑(或直接获取) BibTeX 格式索引文件然后在正文中使用\cs{cite} \cs{citet}等指令
进行引用标注就可以。下面介绍在文章中引用指令的具体使用方法。

\section{总结(顺序编码)}

根据学校要求,参考文献标注用中括号上标形式进行标注。使用方式与效果如下表所展示

\begin{tabular}{l@{\quad$\Rightarrow$\quad}l}
	\verb|\cite{knuth1984texbook}|               & \cite{knuth1984texbook}               \\
	\verb|\citet{knuth1984texbook}|              & \citet{knuth1984texbook}              \\
	\verb|\citep{knuth1984texbook}|              & \citep{knuth1984texbook}              \\
	% 暂不支持
	% \verb|\cite[42]{knuth1984texbook}|           & \cite[42]{knuth1984texbook}           \\
	\verb|\cite{knuth1984texbook,lamport1994latex}| & \cite{knuth1984texbook,lamport1994latex} \\
\end{tabular}

\section{展望(获取BibTeX格式索引)}

获取参考文献的 BibTeX 格式索引有两种方式

\begin{itemize}
	\item 通过Google Scholare或者百度学术等学术文献搜索引擎获取,自行编辑 .bib 文件
	\item 通过Zotero等学术文献整理软件,添加所有的引用文献至库中,导出对应的 .bib 文件
\end{itemize}

编译带参考文献的文章时,我们需要两次编译过程。我们提供了对应的自动化脚本,以及配合vscode latex插件的任务流程,
帮助模板使用者进行编译。

\section{参考文献插入示例}

LaTeX\cite{lamport1994latex}插入参考文献最方便的方式是使用bibliography\cite{pritchard1969statistical},大多数出版商的论文页面\cite{lamport1994latex,pritchard1969statistical}都会有导出bib格式参考文献的链接,把每个文献的bib放入``hutbthesis\_main.bib'',然后用bibkey即可插入参考文献。

潮涌湘江阔,鹏翔天地宽。湖南工商大学正以习近平新时代中国特色社会主义思想为指引,秉持“新工科+新商科+新文科”与理科融合发展的思路,努力形成一流的理念、一流的目标、一流的标准、一流的质量、一流的机制,打造创新工商、人文工商、艺术工商、体育工商、数智工商、绿色工商、幸福工商,建设读书求知的好园地,乘高等教育改革奋进的东风,朝着创新型一流工商大学的愿景扬帆远航。


% % 主文件有代码去掉页眉章节编号的“.”,但这会因为bug导致无编号章节显示一个错误编号,所以这里在无编号章节之前再次重定义sectionmark。
% \renewcommand{\sectionmark}[1]{\markright{#1}}


%%%%%%%%%%%%%%%%%%%%%%%%%%%%%%%%%%%%%%%%%%%%%%%%%%
% 致谢
%
% 存储在\content\acknowledgements.tex文件中
% 根据本科生院的要求,致谢应该在参考文献的前面,不编章号,而附录应该位于参考文献后。
%%%%%%%%%%%%%%%%%%%%%%%%%%%%%%%%%%%%%%%%%%%%%%%%%%
\include{content/acknowledgements}

%%%%%%%%%%%%%%%%%%%%%%%%%%%%%%%%%%%%%%%%%%%%%%%%%%
% 参考文献
%
% 存储在\content\acknowledgements.tex文件中
% 根据本科生院的要求,致谢应该在参考文献的前面,不编章号,而附录应该位于参考文献后。
% 有待修复
%%%%%%%%%%%%%%%%%%%%%%%%%%%%%%%%%%%%%%%%%%%%%%%%%%
% \section{参考文献} % bibliography会自动显示参考文献四个字
\addcontentsline{toc}{chapter}{参考文献} % 由于参考文献不是chapter,这句把参考文献加入目录
% \nocite{*} % 该命令用于显示全部参考文献,即使文中没引用
% cls文件中已经引入package,这里不需要调用 \bibliographystyle 了。
%\bibliographystyle{gbt7714-2005}
%\bibliography{reference}

\printbibliography

%\bibliography{hutbtheisi_main}
\newpage


%%%%%%%%%%%%%%%%%%%%%%%%%%%%%%%%%%%%%%%%%%%%%%%%%%
% 附录部分
%
% 根据学校要求,正文中不应出现长篇幅的代码段或公式推证
% 应单独放置在正文后的附录部分
%%%%%%%%%%%%%%%%%%%%%%%%%%%%%%%%%%%%%%%%%%%%%%%%%%

% https://www.zhihu.com/question/29413517/answer/44358389 %
% 说明如下:
% secnumdepth 这个计数器是 LaTeX 标准文档类用来控制章节编号深度的。
% 在 article 中,这个计数器的值默认是 3,对应的章节命令是 \subsubsection。
% 也就是说,默认情况下,article 将会对 \subsubsection 及其之上的所有章节标题进行编号,也就是 \part, \section, \subsection, \subsubsection。LaTeX 标准文档类中,最大的标题是 \part。它在 book 和 report 类中的层级是「-1」,在 article 类中的层级是「0」。这里,我们在调用 \appendix 的时候将计数器设置为 -2,因此所有的章节命令都不会编号了。不过,一般还是会保留 \part 的编号的。所以在实际使用中,将它设置为 0 就可以了。

% 在修改过程中请注意不要破环命令的完整性

% \renewcommand\appendix{\setcounter{secnumdepth}{-2}}
\appendix
\include{content/appendix}


\end{document}
